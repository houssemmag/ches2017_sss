\documentclass{llncs}
\pagestyle{plain}
\usepackage{graphicx}
\usepackage{hyperref}
\usepackage{amsmath,amssymb}
\usepackage{tabularx}
\usepackage{algorithm}
\usepackage{algpseudocode} 
\renewcommand{\algorithmicrequire}{\textbf{Input:}}
\renewcommand{\algorithmicensure}{\textbf{Output:}}
\usepackage{moreverb}
\usepackage[dvipsnames,table]{xcolor}

\newtheorem{assumption}{Assumption}
\newenvironment{notation}{\begin{trivlist}\item[] {\em Notation.}}%
{\samepage \end{trivlist}}

% Probability definitions

\newcommand\myP[1]{\ensuremath\mathsf{P}\lbrack #1 \rbrack}     % The probability
\newcommand\myE[1]{\ensuremath\mathsf{E}\lbrack #1 \rbrack}     % The expectation
\newcommand\myV[1]{\ensuremath\mathsf{Var}\lbrack #1 \rbrack}   % The variance
\newcommand\mycov[2]{\ensuremath\mathsf{Cov}\lbrack #1;#2 \rbrack}   % The covariance
\newcommand\myI[2]{\ensuremath\mathsf{I}\lbrack #1;#2 \rbrack}  % The mutual information
\newcommand\myH[1]{\ensuremath\mathsf{H}\lbrack #1 \rbrack}     % The entropy
\newcommand\todo[1]{\textcolor{red}{TODO: #1}}
\newcommand{\codel}{$(3,6)\text{-code}$ }

% Finite Fields definitions

\def \GF {\mathrm{GF}}
\def \GFn {\GF(2^n)}
\def \GFm {\GF(2^m)}
\def \nbShares {n}
\def \order {d}
\newcommand{\pubEltNoIndex}[1][i]{\alpha}
\newcommand{\pubElt}[1][i]{\alpha_{#1}}
\newcommand{\pubEltStar}{\alpha^{\star}}
\newcommand{\polyCoef}[1][i]{u_{#1}}
\newcommand{\RS}[1][]{\mathrm{RS}(#1)}
\def \Field {\mathbb{K}}
\def \FField {\mathbb{F}}
\def \pubSet{\mathcal{S}}
\def \myTrace {\mathrm{tr}_{\Field/\FField}}
\def \polySet {\mathcal{P}(\pubEltStar)}

% Editorial definitions

\def\etal{\textit{et al.} }
\newcommand {\ie}{{\em i.e.} }
\def\eg{\textit{e.g.} }
\def\resp{{\em resp.} }
\def\myth{$^\text{th}$}

% ECC
\def\RSCode{\mathcal{C}}
\def\RSProduct{\mathcal{C}^{\star}}
\newcommand{\trans}[1][\RSProduct \rightarrow \RSCode] {\mathrm{\bf transcoding}_{#1}}
\def \wordRS {c}
\def \wordRSNew {\hat{c}}
\def \wordRSProduct {c^{\star}}
\def \wordRSProductZero {c^{0}}


% Comments
\def\comEP#1{\textcolor{RoyalPurple}{{\bf EP: #1}}}

\title{Enhancement of Shamir's Secret Sharing Schemes against Side-Channel Attacks}

\author{Anonymous submission to CHES'2017}
\institute{}

%\author{Houssem MAGHREBI, Emmanuel Prouff}
%\institute{
%SAFRAN Identity & Security, 
%\newline
%18, Chauss\'{e}e Jules C\'{e}sar, 95520 Osny, France.
%\newline
%\email{firstname.lastname@morpho.com}
%}

\begin{document}
\maketitle
\begin{abstract}
 \todo{}
\end{abstract}
{\bf Keywords:} Shamir's secret sharing, repairing codes, masking countermeasures, template attack, side channel attacks.
 

\section{Introduction}
\label{sec-intro}

\subsubsection{Side-Channel Attacks.}
 \todo{}
\subsubsection{Side-Channel Countermeasures.} 
 \todo{}
\subsubsection{Shamir's Secret Sharing based Countermeasures.}

In a seminal paper\,\cite{Sha79}, Shamir proposed to split a secret $A\in \GFn$ into $\nbShares$ shares such that
no tuple of shares with cardinality lower than a so-called {\em threshold} $\order<\nbShares$ depends on $A$. Shamir's
protocol consists in associating $A$ with a random polynomial $P_A(X)\doteq A + \sum_{i=1}^{\order}\polyCoef X^i$ of degree lower than $\order$ and with constant term $A=P_A(0)$. Then, the polynomial $P_A(X)$ is evaluated in $\nbShares$ distinct public non-zero elements $\pubElt[0], \ldots, \pubElt[n-1]$ in $\GFn$ to define a so-called {\em $(\nbShares,\order)$-sharing} $(A_0,A_1,\cdots,A_{\nbShares-1})$ of $A$ such that $(A_i)_{i\leqslant \nbShares}=(P_A(\alpha_i))_{i\leqslant \nbShares}$. To re-construct $A$ from its sharing, polynomial interpolation is first applied to re-construct the polynomial from its $\nbShares$ evaluations $A_i$ and then, it is evaluated in $0$. Actually, using Lagrange's
interpolation formula, the two steps can be combined in a single one thanks to the equality $ A=\sum_{i=0}^{\nbShares-1} A_i \cdot \beta_i\enspace$,  where the (public) constants $\beta_i$ are defined as $\beta_i:=\prod_{k=0,k\neq i}^{\nbShares-1} \frac{\alpha_k}{\alpha_i+\alpha_k}$. As initially observed by McEliece and Sarwate in \cite{MS81}, the sharing of $A$ described above may be viewed as an {\em encoding} with a Reed-Solomon code of parameters $[\nbShares + 1,\order + 1]$:
$$
(A,A_1,\cdots,A_{\nbShares}) = (A,\polyCoef[1],\cdots,\polyCoef[\order])\times 
\left(\begin{matrix}
1 & 1 & \hdots & 1 \\
0 & \pubElt[1] & \hdots & \pubElt[\nbShares] \\
0 & \pubElt[1]^2 & \hdots & \pubElt[\nbShares]^2 \\
\vdots & \vdots & \hdots & \vdots \\
0 & \pubElt[1]^{\order} & \hdots & \pubElt[\nbShares]^{\order} 
\end{matrix}
\right)
$$
The reconstruction of $A$ then corresponds to a simple {\em decoding}.
\vspace{3mm}

To define a $\order$\myth-order masking scheme for a block cipher
implementation where each intermediate result is split with Shamir's
technique, one must specify a secure method for the processing of field
multiplications over $\GFn$. Most of (if not all) existing protocols start from a multiplication scheme introduced by
Ben-Or \etal in the context of the Multy-Party Computation Theory
\cite{BGW88}. For this protocol to work, the number of shares $\nbShares$ per variable must be at least $2\order+1$ and for $\nbShares=2\order+1$, it is proved that it satisfies a security property encompassing the $\order$\myth-order SCA security \cite{PR11}. We give hereafter the adaptation of
\cite{BGW88} in the 
SCA context as proposed in \cite{PR11,RP12}\footnote{The protocol is 
an improved version of the protocol originally proposed by Ben-Or \etal \cite{BGW88},
due to Gennaro \etal in \cite{GRR98}.}.


\begin{algorithm}
\caption{Secure Multiplication For Shamir's Secret Sharing}
\label{alg:secmult}
\begin{algorithmic}[1]
\Require {two integers $\nbShares$ and $\order$ such that $n\geq 2d+1$, the  $(\nbShares,\order)$-sharings $(A_i)_i=(P_A(\pubElt))_i$ and $(B_i)_i=(P_{B}(\pubElt))_i$ of $A$ and $B$ respectively.}
\Require   {the $\nbShares$ distinct points $\pubElt$, the interpolation values $(\beta_0,\cdots,\beta_{\nbShares-1})$}
\Ensure     {the $(\nbShares,\order)$-sharing $(P_{C}(\alpha_i))_i$ of $C=A\cdot B$. }
\For{$i=1$ \textbf{to} $\nbShares$}
\State $W_i \leftarrow P_A(\pubElt)\cdot P_{B}(\pubElt)$

\Comment{Compute a sharing $(Q_i(\alpha_j))_{j\leq d}$ of $W_i$ with $Q_i (X) = W_i + \sum_{j=1}^{\order} a_j\cdot X^j$}
\For{$j=1$ \textbf{to} $\order$}
\State $a_j \leftarrow \texttt{rand}(\GFn)$
\EndFor
\For{$j=1$ \textbf{to} $\nbShares$}
\State $Q_i(\pubElt[j]) \leftarrow W_i +
      \sum_{k=1}^{\order} a_k\cdot \pubElt[j]^k$ \label{step:reshare} 
\EndFor  	
\EndFor
   	
\Comment{Compute the share $C_i = P_{C}(\pubElt)$ for $C=A \cdot B$}
\For{$i=1$ \textbf{to} $\nbShares$}
\State $C_i \leftarrow \sum_{j=1}^{\nbShares} Q_j(\pubElt) \cdot \beta_j$
\EndFor
\Return $(C_i)_i$
\end{algorithmic}
\end{algorithm}



The completeness of Algorithm \ref{alg:secmult} is discussed in \cite{BGW88}. Its $\order$\myth-order SCA security can be straightforwardly deduced from the proof given by Ben-Or \etal in \cite{BGW88} in the secure multi-party computation context. Eventually, for $\nbShares=2\order+1$ (which is the parameter choice which optimizes the security/efficiency overhead), the complexity of Algorithm \ref{alg:secmult} in terms of additions and multiplications is ${\cal O}(\order^3)$.
\subsubsection{Our Contribution.}
 \todo{}
\subsubsection{Paper Outline.}


\section{Exact Repair Problem}
\subsection{Preliminaries}

A {\em linear code} $\mathcal{C}$ of length $\nbShares$ and dimension $k$ over a finite field $\Field$ is a $k$-dimensional subspace of $\Field^{\nbShares}$. It is denoted by $\mathcal{C}[n,k]$. A particular linear code is the {\em Reed-Solomon Code} whose definition is recalled hereafter.

\begin{definition}[Reed-Solomon Code]
The Reed-Solomon code $\RS[\pubSet, \order + 1]\subseteq \Field^{\nbShares}$ of dimension $\order + 1$ over a finite field $\Field$ and with evaluation subset $\pubSet = \{\pubElt[1],\pubElt[2],\cdots,\pubElt[\nbShares]\}$ of $\Field$ is the subspace:
$$\RS[\pubSet,\order + 1] = \{(P(\pubElt[1]),P(\pubElt[2]),\cdots,P(\pubElt[\nbShares])); P(X) \in \Field[X] \text{ and } \deg(P) \leqslant \order\} \enspace .$$
\end{definition}

Reed-Solomon codes are {\em Maximum Distance Separable} (MDS) codes, which means that any $\order + 1$ symbols (that is, evaluations of a polynomial $P(X)\in \RS[\pubSet,\order+1]$) can be used to recover the entire codeword (that is,
$P(X)$ itself).
\vspace{3mm}

Let us assume that a file, consisting of $\order + 1$ blocks, is encoded into $\nbShares$ nodes. {\em Exact Repairing codes} have been originally introduced to recover the contents of an erased node by downloading some information from the remaining nodes. In the language of an $\RS[\pubSet,\order+1]$ code as above, the file is a polynomial $P(X)\in \Field[X]$ with $\deg(P(X)) \leqslant \order$ and each node is associated with an evaluation point $\pubElt \in \pubSet$ and stores $P(\pubElt)$. For an arbitrary $\pubEltStar\in \pubSet$ (corresponding to an erased node), the goal is to recover $P(\pubEltStar)$ given some information from the other nodes. Since the code is MDS, its is straightforward to do it by choosing any subset of $\pubSet$ with cardinality $\order + 1$ and by asking the evaluations $P(\pubElt)$ to the corresponding nodes. This means that, for any $\pubEltStar$, the evaluation $P(\pubEltStar)$ can at least be recovered by asking $\log_2(\Field)$ bits of information (corresponding to $P(\pubElt)$) to $\order +1$ nodes. Linear repairing codes aim to define a repairing scheme such that, for any $\pubEltStar$, the evaluation $P(\pubEltStar)$ can be recovered with strictly less than $(\order + 1)\times \log_2(\Field)$ bits of information, by allowing each node to return any $\FField$-linear function of its content, with $\FField$ being a subfield of $\Field$. The $\FField$-linear functions from $\FField$ to $\Field$ may all be defined from the so-called {\em field trace} of $\Field$ over $\FField$. We recall the definition of the field trace hereafter:

\begin{definition}[Field Trace]
Let $p$ and $r$ be two integers and let $q=p^r$. Let $\Field=\GF(q^t)$ be a degree-$t$ finite field extension of a $\FField=\GF(q)$. The {\em field trace} $\myTrace$ is defined for every $\beta \in \Field$ as:
\begin{equation}\label{equ:trace}
\myTrace(\beta) = \beta + \beta^q + \beta^{q^2} + \cdots + \beta^{q^{t-1}} \enspace .
\end{equation}
\end{definition}

In a linear repair scheme, the node corresponding to a public point $\pubElt$ returns zero or more elements of $\FField$ of the form $\myTrace(\gamma P(\pubElt))$ for some field elements $\gamma \in \Field$. A linear exact repair scheme can then be described by the field elements $\gamma$ used by all the nodes, along with a (linear) repair algorithm. A precise definition in the context of Reed-Solomon codes is given below:

\begin{definition}[Linear Exact Repair Scheme (LERS)]\label{def:LERS}
Let $\pubSet = \{\pubElt[1],\pubElt[2],\cdots,\pubElt[\nbShares]\}$ be an evaluation subset of a finite field $\Field$ and let $\RS[\pubSet, \order + 1]\subseteq \Field^{\nbShares}$ be the corresponding Reed-Solomon code. A {\em linear exact repair scheme} for this code over a subfield $\FField$ of $\Field$ is composed of the following steps:
\begin{itemize}
\item for each $\pubEltStar\in \pubSet$, and for each $\pubEltNoIndex\in \pubSet \slash \{\pubEltStar\}$, a set of queries $Q_{\pubEltNoIndex}(\pubEltStar)\subseteq \Field$,
\item for each $\pubEltNoIndex \in \pubSet$ and any polynomial $P(X)\in \Field[X]$ with degree lower than $\order +1$, a linear reconstruction algorithm that computes 
$$
P(\pubEltStar) = \sum_{i=1}^t\lambda_i \cdot \nu_i \enspace ,
$$
for coefficients $\lambda_i\in \FField$ and a basis $\{\nu_1,\nu_2,\cdots,\nu_t\}$ for $\Field$ over $\FField$, so that the coefficients $\lambda_i$ are $\FField$-linear combinations of the queries in 
$$
\bigcup_{\pubEltNoIndex \in \pubSet \slash \{\pubEltStar\}}\{\myTrace(\gamma\cdot P(\pubEltNoIndex); \gamma \in Q_{\pubEltNoIndex}(\pubEltStar)\} \enspace .
$$
\end{itemize}
\end{definition}

The {\em repair bandwith} $RB$ of a LERS is the maximum number of sub-symbols which must be returned by the nodes to recover $P(\pubEltStar)$. It is defined by:
$$
RB = \max_{\pubEltStar \in \pubSet} \sum_{\pubEltNoIndex \in \pubSet \slash \{\pubEltStar\}} \# Q_{\pubEltNoIndex}(\pubEltStar) \enspace .
$$

One of the main goals of LERS schemes is to minimize the maximum number of bits $RB\times \log_2(\FField)$ required to reconstruct $P(\pubEltStar)$ for any $\pubEltStar\in \pubSet$ and any polynomial $P(X)$. During the last few years, there have been several publications tackling this issue in the context of RS codes (see \eg \cite{DRWS11,SPDC13}), or more generally, for MDS codes (see \eg \cite{CHJL11,DGWWR10,SR10}). Depending on the extension dimension $t$ of $\Field$ over $\FField$, some lower bounds have been given:
\begin{itemize}
\item when $t$ is sufficiently large, $RB$ is bounded below by $(t\cdot RL)/(RL-\order)$, where $RL$ denotes the {\em repair locality} and corresponds to the maximum number of $\pubElt \in \pubSet$ that are required to recover $P(\pubEltStar)$ whatever $\pubEltStar \in \pubSet$. For very large values of $t$, several schemes have been proposed for which the lower bound is achieved (see \eg \cite{CHJL11,SR10}).
\item when $t$ is small, it is clear that the lower bound above cannot be achieved since we have $RB \geqslant \order + t$. It must however be observed that this latter bound is much better than the naive one $t\times (\order + 1)$. \comEP{To continue...}
\end{itemize}
\subsection{Explicit Constructions for Reed-Solomon Codes}
In the recent work \cite{GW16}, the authors show that the construction of a LERS for a Reed-Solomon code $\RS[\pubSet, \order + 1]$ and a subfield $\FField$ of $\Field$
is equivalent to the finding, for any $\pubEltStar\in \pubSet$, of field elements $(\zeta_j)_{j\leqslant t}$ and $(\mu_{\pubEltNoIndex,\zeta_j}(\pubEltStar))_{j\leqslant t,,\pubEltNoIndex \in \pubSet}$, such that:
\begin{itemize}
\item (1) for any $j\in [1:t]$, we have 
\begin{equation}\label{equ:LERS_1}
\myTrace(\zeta_j P(\pubEltStar)) = \sum_{\pubEltNoIndex \in \pubSet \slash \{\pubEltStar\}} \myTrace(\mu_{\pubEltNoIndex,\zeta_j}(\pubEltStar)P(\pubEltNoIndex))
\end{equation}
\item and (2) the following equality holds
\begin{equation}\label{equ:LERS_2}
P(\pubEltStar) = \sum_{j=1}^t \myTrace(\zeta_j P(\pubEltStar)) \enspace .
\end{equation}
\end{itemize}
The core observation in \cite{GW16} is that, surprisingly, for some $\pubEltNoIndex\in \pubSet$ the size of all the elements required to build the set $\{\myTrace(\zeta_j P(\pubEltNoIndex));j\in [1;t]\}$ may be smaller than $P(\pubEltNoIndex)$. This implies that the total number of bits required to recover $P(\pubEltStar)$ from \eqref{equ:LERS_1} and \eqref{equ:LERS_2} may be smaller than $(\order +1)\log_2(\# \Field)$ (corresponding to a direct polynomial interpolation). Actually, it is shown in \cite{GW16} that the construction of field elements $(\zeta_j)_{j\leqslant t}$ and $(\mu_{\pubEltNoIndex,\zeta_j}(\pubEltStar))_{j\leqslant t,,\pubEltNoIndex \in \pubSet}$ minimizing the repair bandwidth is equivalent to the problem of finding some nice polynomials over $\Field$ \cite[Theorem 4]{GW16} and some explicit constructions are proposed. In the specific case of Reed-Solomon codes $\RS[\pubSet, \order + 1]$, the construction of the coefficients $(\zeta_j)_{j\leqslant t}$ and $(\mu_{\pubEltNoIndex,\zeta_j}(\pubEltStar))_{j\leqslant t,,\pubEltNoIndex \in \pubSet}$ leading to some repair bandwidth $RB$ over a subfield $\FField\subseteq \Field$ such that $t=[\Field:\FField]$ can be directly deduced from a set $\polySet\in \Field[X]$ of $t$ polynomials of degree $\nbShares - \order - 1$ so that:
$$
\dim_{\FField} (\{p(\pubEltStar): p\in \polySet\}) = t
$$
and the sets $\{p(\pubEltNoIndex): p\in \polySet\}$ for $\pubEltNoIndex \neq \pubEltStar$ satisfy:
$$
RB \geqslant \sum_{\pubEltNoIndex \in \pubSet \backslash \{\pubEltStar\}} \dim_{\FField} (\{p(\pubEltNoIndex): p\in \polySet\}) \enspace .
$$
Once such a family of polynomials $\polySet$ is found the coefficients are built so that
$$
\{\zeta_1,\cdots,\zeta_t\} = \{p(\pubEltStar): p\in \polySet\}
$$
and for every $j\in [1:t]$ and $\pubEltNoIndex \in \pubSet \backslash \{\pubEltStar\}$;
$$
\mu_{\pubEltNoIndex,\zeta_j}(\pubEltStar) = p(\pubEltNoIndex)\times \frac{\prod_{\beta\in \pubSet \backslash \{\pubEltStar\}}(\pubEltStar - \beta)}{\prod_{\beta\in \pubSet \backslash \{\pubEltStar\}}(\pubEltNoIndex - \beta)} \enspace .
$$

 
\subsection{Example}
 \todo{}

\section{Attack against Shamir's Secret Sharing Scheme }
\subsection{Attack Description}
 \todo{}
\subsection{Attack results on Simulated Traces}
 \todo{}
\subsection{Attack results on Real Device}
 \todo{}


\section{Improvement Proposal of Shamir's Secret Sharing Scheme}
 \todo{}

\section{Conclusion}
\label{sec-conc}
 \todo{}
 
\bibliographystyle{abbrv}
\bibliography{sca,ANSSI}

\end{document}
